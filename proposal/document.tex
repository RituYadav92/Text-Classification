% TODO: enable english (?)
% TODO: GANTT-chart (in appendix)

% \documentclass[10pt, oneside, english]{article}   	
\documentclass[10pt, oneside]{article}
\usepackage{geometry}                		
\geometry{a4paper}                   		
% \usepackage[english, es-noindentfirst]{babel}
% \selectlanguage{english}
\usepackage[utf8]{inputenc}               		
\usepackage{graphicx}			
\usepackage{amssymb}
\usepackage{authblk}
\usepackage{multicols}


\title{Word embeddings for predicting political affiliation based on twitter data}
\author[]{IBRAHIM ABDELAZIZ}
\author[]{OLIVER BERG}
\author[]{ANGJELA DAVITKOVA}
\author[]{MD ASHRAF HOSSAIN}
\author[]{CHARU JAMES}
\author[]{SHRIRAM SELVAKUMAR}
\author[]{KUMAR SHRIDHAR}
\author[]{SAURABH VARSHNEYA}
\affil[1]{Technische Universität Kaiserslautern}




\begin{document}
\maketitle
\begin{multicols}{2}


\section{Introduction}

The modern world of social media knows a plethora of means to communicate ones personal opinion and political alignment. With the platform \textit{Twitter}, figures of political interest are expressing their standpoints in small-sized 144-character texts, which contain a comprised message specific to the general public. This yields great potential for automated analysis of party affiliations to classify political persons of interest within the overall political spectrum \cite{Biessmann2017}.

Hence do we motivate the structured approach of building a \textbf{social media dataset}, utilizing \textbf{word embeddings} \cite{Pelevinala2016} based modeling approaches to prepare further \textbf{qualitative analysis} to obtain dedicated insight and validate possible early intuition.
This is to be seen in context of latest \textbf{advances in research}.


\section{Data assets}

According to the presented task description of analyzing political affiliation based on Twitter-data, the final comparison occurs on the basis of \textbf{tweets} made by political figures on the platform Twitter.

Additionally, categorized data taken from "www.wahl.de/politiker" is leveraged as a prearrangement of the initial raw Twitter-data.


\section{Model Deployment}

[.. generally: All word embedding stuff ..]

[.. which algorithms / categories of algorithms to test first/second/third ..]

[.. which approaches may yield which types of results; generally: what is to be expected? ..]


\section{Analysis of Results}

[.. Quantitative Analysis ..]


\section{Final Research Questions}

[.. Questions that came up during implementation, modeling and evaluation ..]


\bibliography{lit}
\bibliographystyle{plain}
\end{multicols}
\end{document}   

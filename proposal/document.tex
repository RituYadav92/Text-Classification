% TODO: enable english
% TODO: add proper todo package
% TODO: proper literature listing!!! (paper listed in task, papers to methods, etc.)
% TODO: GANTT-chart
% TODO: incorporate proper appendix for a) GANTT-chart, b) literature listing

% \documentclass[10pt, oneside, english]{article}   	
\documentclass[10pt, oneside]{article}
\usepackage{geometry}                		
\geometry{a4paper}                   		
% \usepackage[english, es-noindentfirst]{babel}
% \selectlanguage{english}
\usepackage[utf8]{inputenc}               		
\usepackage{graphicx}			
\usepackage{amssymb}
\usepackage{authblk}


\title{Word embeddings for predicting political affiliation based on twitter data}
\author[]{IBRAHIM ABDELAZIZ}
\author[]{OLIVER BERG}
\author[]{ANGJELA DAVITKOVA}
\author[]{MD ASHRAF HOSSAIN}
\author[]{CHARU JAMES}
\author[]{SHRIRAM SELVAKUMAR}
\author[]{KUMAR SHRIDHAR}
\author[]{SAURABH VARSHNEYA}
\affil[1]{Technische Universität Kaiserslautern}
% \renewcommand\Authands{, }
% \date{}	note: this only works if commented!!(???)




\begin{document}

\maketitle



\section{Investigated Problem}

The modern world of social media knows a plethora of means to communicate ones personal opinion and political alignment. With the platform \textit{Twitter}, figures of political interest are expressing their standpoints in small-sized 144-character texts, which contain a comprised message specific to the general public. This yields great potential for automated analysis of party affiliations to classify political persons of interest within the political spectrum.

[.. \#taskDescription \& why do this in the first place ..]



\section{Literature Research}

A thorough introduction to political affiliation analysis was provided in \textit{"Predicting political party affiliation from text" by Biessmann et. al.}, where political motives were shown to be consistently predictable with an accuracy better than chance. 

[.. papers for methods deployed in example paper and/or our later approaches ..]



\section{Considerations towards Data}

According to the presented task description of analyzing political affiliation based on Twitter-data, the final comparison occurs on the basis of \textbf{tweets} made by political figures on the platform Twitter.

Along the ground work established in \textit{"Predicting political party affiliation from text" by Biessmann et. al.}, the learning models are also deployed against \textbf{parliament discussion data} as well as \textbf{party manifesto data} to establish a categorical groundwork.

[.. more specifically: which data should be used how ..]

[...]



\section{Deployed Methods}

[.. which algorithms / categories of algorithms to test first/second/third ..]

[.. which approaches may yield which types of results; generally: what is to be expected? ..]



\end{document}   
